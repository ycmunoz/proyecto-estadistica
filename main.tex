
\documentclass[12pt,a4paper]{article}
\usepackage[a4paper, total={6.7in,8.1in}]{geometry}
\usepackage{color}
%\usepackage{geometry}
%\geometry{legalpaper, landscape, margin=2in}
\usepackage[utf8]{inputenc}
\usepackage{multicol}
\setlength{\columnsep}{1cm}
\title{Aproximaciones Gausssianas:\\Normal, Binomial y Poisson}
\author{Condori Muñoz Rommel Yoshimar\\Abel\\Rodrigo}
\date{September 2018}

\begin{document}

\maketitle
\begin{multicols}{2}
\section{Introducción}
    
    En ocasiones, algunas variables aleatorias siguen distribuciones de probabilidad muy concretas, como por ejemplo el estudio a un colectivo numeroso de individuos que se modelizan por la distribución “Normal”. Veremos sólo algunas de las distribuciones o modelos de probabilidad más importantes y de éstos las aproximaciones Gaussianas que se dan para las distribuciones Normal, Binomial y Poisson   y que después nos resultarán muy útiles para el tema de la Estimación.
    Como hemos visto, las variables pueden ser discretas o continuas; por ello, también las distribuciones a tratar podrán ir asociadas a variables aleatorias discretas o continuas.\\
    
    \\\\ \textbf{Distribución Binomial}, que es una extensión de Bernouilli. Supongamos que se repite un experimento "n" veces  de forma idéntica e independiente. Los resultados  de cada  realización  del experimento  se clasifican  en dos categorías, una será la probabilidad de éxito,  y otra de fracaso. Así, por tanto, una variable aleatoria discreta X se distribuye como una Distribución Binomial  de dos parámetros.
    \\ \textbf{Distribución Poisson}, que es una distribución discreta de gran utilidad  sobre todo en procesos biológicos, donde X suele representar el número  de eventos independientes que ocurren a velocidad constante en un intervalo de tiempo o en un espacio.
    \\ \textbf{Distribución Normal}, es una de las distribuciones más utilizados  en la práctica, ya que  multitud  de fénomenos se comportan según  una distribución normal. Esta distribución se caracteriza porque los valores  se distribuyen  formando una campana de Gauss, en torno  a un valor central que coincide con el valor  medio de la distribución. Las ventajas teóricas de este modelo hacen que su uso se distribuye como una normal.
    [1] [2]
    
    En el presente proyecto se verá el comportamiento de éstas aproximaciones gaussianas en una simulación  que se ha de dar por medio de R Studio y así concluir que tan importante es en nuestro entorno.
    
\section{Estado del arte}
    \begin{itemize}
        \item \textbf{Clasificación de eventos sísmicos empleados  procesos gaussianos}: La clasificación de señales es de crucial  importancia para el descubrimiento de posibles interacciones entre movimientos telúricos volcánicos per se. En este artículo  se presenta la aplicación de procesos gaussianos para la clasificación de eventos sísmicos registrados en un nevado en partilar.Las señales se caracterizan usando los coeficientes  de un modelo autoregresivo, empleado para estimar la densidad espectral de potencia. La función de distribución predictiva para la clasificación se aproxima mediante el método de Laplace.El desempeño obtenido es mayor que el de una red neuronal artificial, clasificador  utilizado tradicionalmente para resolver una tarea.[3]\\
        \item \textbf{Discontinuidad en la BMV: Aplicando Procesos Poisson-Gaussianos a los Activos Nacionales. Desechando la Distribución Normal}:La administración de riesgos actual se divide en tres grandes temas: el cálculo de productos derivados, la modelación de las tasas de interés y el área de riesgos financieros y económicos.Específicamente, desde los trabajos realizados por Bachelier(1900), la modelación financiera ha involucrado la presencia delmovimiento Browniano. Lo anterior nos conduce a mantener supuestos que incluyen desde comportamientos log normales por parte de los rendimientos de los activos hasta varianzas que no son proporcionales al tiempo. Este trabajo propone el uso de una distribución diferente a la distribución normal para la teoría financiera utilizando los rendimientos de un grupo de activos nacionales. Se trata del uso del modelo Poisson-Gaussiano. Se aplica una aproximación propuesta por Sanjiv Das (1998) en la obtención de la función de verosimilitud para el caso de once activos pertenecientes a la BMV y sus series correspondientes del10 de enero del 1994 al 31 de diciembre del 2004.[4]\\
        \item \textbf{Aproximación mediante Gaussianas  de datos electroforéticos}: La electroforesis capilar (EC) es una técnica de separación y análisis de sustancias químicas ampliamente utilizada en la industria biotecnológica y bioquímica. El resultado del análisis de una muestra química con EC es una señal llamada electroferograma donde varios picos  representan los distintos subcomponentes de la muestra. La forma de cada pico, bajo condiciones determinadas, puede modelarse con una función gaussiana, aunque  frecuentemente los picos  pueden presentar importantes deformaciones en su forma ideal debido a procesos físico-químicos que ocurren  dentro del capilar. Estas formas  no son exactamente  gaussianas pueden ser modeladas con otras funciones que llamamos gaussianas modificadas. El objetivo del trabajo  fue la obtención  de los parámetros que definen a cada gaussiana, modelando a su vez  las prolongaciones de los picos ("tailing") mediante gaussianas modificadas. Se realiza un proceso de análisis  previo mediante transformada waveleis discreta para las disminución de ruido y la reducción de la dimensión de los datos, y adicionalmente se aplica un algoritmo de corrección de línea base. Se calculan los parámetros iniciales de las gaussianas (ubicación, amplitud, ancho) y finalmente se realiza la aproximación  definitiva de la  curva compuesta  por sumas gaussianas  a la señal original  mediante un proceso  de optimización no lineal (región de confianza). El beneficio  práctico  de la descomposición en suma de gaussianas  de la señal electroforética se aprecia de manera relevante en cuanto a la  significativa reducción en la cantidad de datos  al 0.47\%, ideal para manejar los sistemas emergentes de recolección  de muestras químicas de alta resolución en tiempo y de electroforesis multicapilar, los cuales generan grandes cantidades de  datos en muy poco tiempo.[5]
    \end{itemize}
\section{Bibliografía}
\begin{enumerate}
    \item Montero Alonso, Miguel Ángel, (2007), Apuntes de Estadística II, Granada-España, Editorial Universidad de Granada , pag. 33-44.
    \item Robert V. Hogg, Allen Craig, Joseph W. Mackean,2004 Introductions to Mathematical Stadistics. 6ta Ed., Editorial Pearson, pag 133-160. 
    \item Alvarez,M ,Henao,R ,Duque,E (2007) , Scientia et Technica, No.35, pag. 145-150.
    \item Moreno Quezada, Guillermo E., (02-09-2016), Discontinuidad en la BMV: Aplicando Procesos Poisson-Gaussianos a los Activos Nacionales. Desechando la Distribución Normal., Instituto Tecnológico y de Estudios Superiores de Monterrey, Monterrey- México.
    \item Ceballos,Gerardo A., (2010), Aproximación mediante Gaussianas de datos electroforéticos, Universidad de Los Andes, Mérida-Venezuela.
\end{enumerate}
\end{multicols}
\end{document}
